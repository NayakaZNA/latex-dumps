\documentclass{article}
\pagenumbering{gobble}

% Packages
\usepackage[margin=1.25cm]{geometry} 
\usepackage[fontsize=10pt]{fontsize}

%% Math and Physics
\usepackage{amsmath}
\usepackage{physics}

%% Image
\usepackage{graphicx}

%% Formatting
\usepackage{multicol}
\usepackage{indentfirst}
\usepackage{titlesec}
\usepackage[dvipsnames]{xcolor}
\usepackage{hyperref}
\hypersetup{
	colorlinks=true,
	linkcolor={red!50!black},
	citecolor={blue!50!black},
	urlcolor={blue!50!black}
}

% Settings
\renewcommand*{\thefootnote}{[\arabic{footnote}]}
\renewcommand\thesection{\Roman{section}}
\titleformat{\section}
{\sffamily\large\scshape}
{\thesection}{.5em}{}

% Contents
\begin{document}
	\noindent Z. Nayaka Athadiansyah \hfill {\Large Asal-Muasal $v_{rms}$} \hfill FI1101-25 STEI\\
	19623116 \hfill {} \hfill 13/11/2023
	
	\begin{multicols}{2}    
		\section{Pendahuluan}
		Dalam statistika, akar rerata kuadrat (\textit{root mean square}, RMS) dari $n$ buah data diskrit $x_i (1 \leq i \leq n$) adalah akar dari rata-rata kuadrat data, yakni\footnote{Weisstein, Eric W. "Root-Mean-Square." Dari \textit{MathWorld--A Wolfram Web Resource}. Diakses melalui \href{https://mathworld.wolfram.com/Root-Mean-Square.html}{https://mathworld.wolfram.com/Root-Mean-Square.html} \label{fn2}}
		
		\vspace{-1.25em}
		\begin{equation}
			x_{rms} = \sqrt{\frac{x_1^2 + x_2^2 + \ldots + x_n^2}{n}} = \sqrt{\frac{1}{n} \sum_{i=1}^{n} x_i^2} = \sqrt{\expval{x^2}}
		\end{equation}
		
		Adapun definisi RMS untuk data yang terdistribusi secara kontinu menurut fungsi $f(x)$ ($D_f = [X_1, X_2]$) adalah\textsuperscript{\ref{fn2}}
		
		\begin{equation}
			x_{rms} = \sqrt{\frac{1}{X_2 - X_1} \int_{X_1}^{X_2} f(x)^2 \dd{x}}
		\end{equation}
		\vspace{-.5em}
		
		Konsep \textit{standar deviasi} yang dikembangkan oleh Karl Pearson, Carl Friedrich Gauss, Sir George Biddel Airy, dan matematikawan lainnya juga berasal dari RMS.\footnote{Taleb, Nasim N. (2014). "Standard Deviation". Dari \textit{2014: What Scientific Idea is Ready for Retirement?}. Edge.org. Diakses melalui \href{https://www.edge.org/response-detail/25401}{https://www.edge.org/response-detail/25401}} RMS juga digunakan di fisika, misalnya untuk menyatakan nilai efektif daya, potensial, dan kuat arus rangkaian AC. 
		\par Barangkali yang menjadi fokus pembahasan ini adalah RMS untuk kecepatan partikel dalam teori kinetik gas (TKG), yakni $v_{rms}$. \textit{Mengapa ia tiba-tiba muncul dan mengapa kita gunakan RMS ketimbang rata-rata aritmetika?}
		
		\section{Pembahasan}
		\textbf{Pertama.} Menurut asumsi dasar TKG, gas kinetik dalam kondisi stabil tersusun atas molekul-molekul identik dengan jumlah banyak yang selalu bergerak secara acak menurut hukum Newton.\footnote{Sears, Francis W. dan Salinger, Gerhard L. (1982). \textit{Thermodynamics, Kinetic Theory, and Statistical Thermodynamics}. Canada: Addison-Wesley.\label{fn3}}\textsuperscript{\ref{fn4}}
		
		\par Akibatnya, untuk setiap partikel, kita bisa temukan satu partikel lain yang bergerak ke arah berlawanan sehingga vektor kecepatan mereka saling meniadakan sehingga kecepatan rata-rata seluruh partikel nol dan ini tidak begitu membantu kita memahami kondisi partikel-partikel dalam gas. \\
		
		\par 
		\textbf{Kedua.} Barangkali argumen matematis dapat menjelaskan dengan baik dari mana $v_{rms}$ muncul. Perhatikan bahwa energi kinetik rata-rata seluruh partikel dalam gas dapat dinyatakan sebagai

		\vspace{-1.5em}
		\begin{align}
			\allowdisplaybreaks
			\expval{K} &= \frac{1}{n} \sum_{i=1}^{n} K_i = \frac{1}{n} \sum_{i=1}^{n} \qty(\frac{1}{2} m_iv_i^2) = \frac{1}{2} m \qty(\frac{1}{n} \sum_{i=1}^{n} v_i^2) \notag \\[.5em]
			&= \frac{1}{2} m \expval{v^2} \label{k_ek}
		\end{align}
		
		\par Di samping itu, energi kinetik rata-rata gas juga dideskripsikan oleh persamaan\footnote{Jeans, James. (1967). \textit{An Introduction to the Kinetic Theory of Gases}. New York, NY: Cambridge University Press.\label{fn4}}\footnote{Ling, Samuel J.; Moebs, William; Sanny, Jeff. (2016). \textit{University Physics Volume 2.} Houston, TX. Diakses melalui \href{https://openstax.org/books/university-physics-volume-2/pages/2-2-pressure-temperature-and-rms-speed}{https://openstax.org/books/university-physics-volume-2/pages/2-2-pressure-temperature-and-rms-speed}\label{fn5}}
		
		\vspace{-1.5em}
		\begin{equation}
			\expval{K} = \frac{3}{2}k_BT \label{k_tkg}
		\end{equation}
		
		dengan $k_B = 1{,}380649 \times 10^{-23} \ \mathrm{m}^2 \cdot \mathrm{kg} \cdot \mathrm{s}^{-2} \cdot \mathrm{K}^{-1}$ adalah konstanta Boltzmann dan $T$ adalah temperatur absolutnya. Dari Persamaan \ref{k_ek} dan \ref{k_tkg} didapatkan\textsuperscript{\ref{fn3}}\textsuperscript{\ref{fn5}}
		
		\vspace{-.5em}
		\begin{equation}
			\sqrt{\expval{v^2}} = v_{rms} = \sqrt{\frac{3k_BT}{m}}
		\end{equation}
		
		Kecepatan ini adalah kecepatan yang dimiliki partikel dalam gas jika seluruh partikel diberi energi kinetik secara sama rata dengan besar $3k_B T/2$. \\
		
		\textbf{Terakhir.} Ketimbang menghitung kecepatan rata-rata, mengapa kita tidak hitung \textit{kelajuan} rata-ratanya ($\expval{|v|}$) saja?
		
		Kita lebih tertarik dengan $v_{rms}$ ketimbang $\expval{|v|}$ karena $v^2$ berkaitan erat dengan energi kinetik $K$ (lihat Pers. \ref{k_ek}). Lebih lanjut lagi, $K$ juga berhubungan dengan temperatur gas menurut Persamaan \ref{k_tkg}. Jadi, kuantitas $v_{rms}$ lebih relevan
		
		\par Sebagai tambahan, grafik distribusi Maxwell-Boltzmann berikut menunjukkan sebaran probabilitas kecepatan partikel dalam gas ideal:
		
		{
		\vspace{.5em}
		\centering
		\includegraphics[scale=0.13]{diagram.png}
		\par
		}
		
		\section{Pertanyaan}
		\begin{enumerate}
			\item \textbf{Bagaimana} $v_{rms}$ berubah ketika partikel gas mendekati kecepatan cahaya, dan \textbf{apa} saja prinsip relativitas yang perlu kita perhitungkan?
			\item Jika gas dikenai medan eksternal---misalnya medan magnetik atau medan listrik, \textbf{apa} pengaruhnya terhadap $v_{rms}$ dan distribusi partikel-partikel gas? \textbf{Apa} teknologi yang mungkin bisa dirancang berdasarkan hal ini?
		\end{enumerate}
	\end{multicols}
	
\end{document}
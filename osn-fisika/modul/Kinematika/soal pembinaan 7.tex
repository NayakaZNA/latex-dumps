\documentclass[12pt, a4paper]{article}\usepackage[utf8]{inputenc}
\usepackage[paperwidth=210mm, paperheight=297mm, margin=1.5cm]{geometry}
\usepackage[svgnames]{xcolor}
\usepackage[T1]{fontenc}
\usepackage{arabtex}
\usepackage{utf8}
\setcode{utf8}
\usepackage{lipsum}
\usepackage[T1]{fontenc}
\usepackage{fouriernc}
%\usepackage{kpfonts, baskervald}
\usepackage{amsmath, systeme}
\usepackage{amssymb}
\usepackage{spalign}
\usepackage{relsize}
\usepackage{color}
\usepackage{amsthm}
\usepackage{subfig}
\usepackage{cancel}
\usepackage{xstring}
\usepackage{tikz}
\usepackage{cases}
\usepackage{showlabels}
\usepackage{multicol}
\usetikzlibrary{arrows}
\usepackage[shortlabels]{enumitem}
\usepackage{varwidth}
\usepackage{mathtools}
\usepackage{graphicx}
\usepackage{verbatim}
\usepackage{anyfontsize}
\usepackage[fontsize=11pt]{fontsize}
\usepackage{array}
\usepackage{cite}
\usepackage{titling}
%\usepackage[toc]{multitoc}


\setlength{\droptitle}{-5em}


\newcommand{\defeq}{\vcentcolon=}

\newcommand{\eqdef}{=\vcentcolon}

\newcommand\hcancel[2][merah]{\setbox0=\hbox{$#2$}%
	\rlap{\raisebox{.45\ht0}{\textcolor{#1}{\rule{\wd0}{1pt}}}}#2} 

\newcommand{\garis} [3] []{
	\begin{center}
		\begin{tikzpicture}
			\draw[#2-#3, ultra thick, #1] (0,0) to (1\linewidth,0);
		\end{tikzpicture}
	\end{center}
}

\newcommand{\chapternote}[1]{{%
		\let\thempfn\relax% Remove footnote number printing mechanism
		\footnotetext[0]{\emph{#1}}% Print footnote text
}}

\renewcommand{\figurename}{Gambar}

\renewcommand{\thesection}{\Alph{section}} 
\renewcommand{\thesubsection}{\thesection.\arabic{subsection}}
\renewcommand{\thesubsubsection}{\thesection.\arabic{subsubsection}.}

\renewcommand*\contentsname{Daftar Isi}

\renewcommand{\thefootnote}{\roman{footnote}}

\newcommand*{\coret}[1]{\renewcommand{\CancelColor}{\color{#1}}\cancel}

\newcommand\Ccancel[2][black]{
	\let\OldcancelColor\CancelColor
	\renewcommand\CancelColor{\color{#1}}
	\cancel{#2}
	\renewcommand\CancelColor{\OldcancelColor}
}

\theoremstyle{definition}
\newtheorem{definisi}{Definisi}

\theoremstyle{definition}
\newtheorem{teorema}{Teorema}


%\renewcommand*{\multicolumntoc}{2}
%\setlength{\columnseprule}{0.5pt}


\usepackage[colorlinks=true, linkcolor=black, urlcolor=black]{hyperref}

\begin{document}
	
	\begin{center}
		\textbf{{\LARGErrr
				Kinematika (GLBB)\\[13pt]} {\large B-Bolt Fisika SMA Negeri 3 Malang}}
		\vspace{-.5em}
		\garis{diamond}{diamond}
	\end{center} 
	
	\begin{enumerate}
		\item $[$8 poin$]$ \textbf{Perjalanan Singkat Sebuah Mobil.} Sebuah mobil bergerak lurus dari keadaan diam, mula-mula dengan percepatan $a = 5{,}0$ m/s$^2$, lalu bergerak dengan kecepatan konstan beberapa saat, lalu mengalami perlambatan sebesar $a$ hingga berhenti. Waktu total pergerakan mobil adalah $t = 25$ detik. Kecepatan rata-rata sepanjang selang waktu tersebut adalah $v = 72$ km/jam. Berapa lama mobil tersebut bergerak secara seragam? [Irodov 1.3]
		
		\vspace{15em}
		
		\item Sebuah mobil dipercepat dari keadaan diam dengan percepatan $\alpha$. Setelah itu mobil diperlambat dengan perlambatan $\beta$ hingga berhenti. Total waktu yang dibutuhkan adalah $T$ detik. Berapa jarak yang ditempuh mobil ini?
		
		\vspace{15em}
		
		\item Dua kereta, $A$ dan $B$, bergerak di arah yang sama dalam jalur kereta yang sama. Mula-mula, $A$ diam pada posisi $d$ dan $B$ bergerak dengan kecepatan $v_0$ di titik asal. $A$ bergerak dengan percepatan $a$ sedangkan $B$ bergerak dengan percepatan $-a$. Tentukan nilai maksimum $v_0$ (dinyatakan dalam $d$ dan $a$) supaya kedua kereta tidak bertabrakan?
		
		\pagebreak
		
		\item Sebuah bola dijatuhkan dari keadaan diam pada ketinggian $h$. Tepat di bawahnya dari tanah, bola kedua secara bersamaan dilemparkan ke atas dengan kecepatan $v_0$. Jika kedua bola bertumbukan ketika bola kedua diam sesaat, pada ketinggian berapa tumbukan terjadi?
		
		\vspace{15em}
		
		\item Sebuah bola dijatuhkan tanpa kecepatan awal dari ketinggian $4h$. Setelah bola jatuh sejauh $d$, bola kedua dijatuhkan tanpa kecepatan awal dari ketinggian $h$. Berapa nilai $d$ (dinyatakan dalam $h$) supaya kedua bola menumbuk tanah bersamaan? [Morin 2.9]
	\end{enumerate}
	
\end{document}

